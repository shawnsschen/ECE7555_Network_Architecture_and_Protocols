%%%%%%%%%%%%%%%%%%%%%%%%%%%%%%%%%%%%%%%%%
% Short Sectioned Assignment
% LaTeX Template
% Version 1.0 (5/5/12)
%
% This template has been downloaded from:
% http://www.LaTeXTemplates.com
%
% Original author:
% Frits Wenneker (http://www.howtotex.com)
%
% License:
% CC BY-NC-SA 3.0 (http://creativecommons.org/licenses/by-nc-sa/3.0/)
%
%%%%%%%%%%%%%%%%%%%%%%%%%%%%%%%%%%%%%%%%%

%----------------------------------------------------------------------------------------
%	PACKAGES AND OTHER DOCUMENT CONFIGURATIONS
%----------------------------------------------------------------------------------------

\documentclass[titlepage, paper=a4, fontsize=11pt]{scrartcl} % A4 paper and 11pt font size

\usepackage[T1]{fontenc} % Use 8-bit encoding that has 256 glyphs
\usepackage{fourier} % Use the Adobe Utopia font for the document - comment this line to return to the LaTeX default
\usepackage[english]{babel} % English language/hyphenation
\usepackage{amsmath,amsfonts,amsthm} % Math packages
\usepackage{listings}

\usepackage{lipsum} % Used for inserting dummy 'Lorem ipsum' text into the template


\usepackage{sectsty} % Allows customizing section commands
\allsectionsfont{\centering \normalfont\scshape} % Make all sections centered, the default font and small caps

\usepackage{fancyhdr} % Custom headers and footers
\pagestyle{fancyplain} % Makes all pages in the document conform to the custom headers and footers
\fancyhead{} % No page header - if you want one, create it in the same way as the footers below
\fancyfoot[L]{} % Empty left footer
\fancyfoot[C]{} % Empty center footer
\fancyfoot[R]{\thepage} % Page numbering for right footer
\renewcommand{\headrulewidth}{0pt} % Remove header underlines
\renewcommand{\footrulewidth}{0pt} % Remove footer underlines
\setlength{\headheight}{13.6pt} % Customize the height of the header

\numberwithin{equation}{section} % Number equations within sections (i.e. 1.1, 1.2, 2.1, 2.2 instead of 1, 2, 3, 4)
\numberwithin{figure}{section} % Number figures within sections (i.e. 1.1, 1.2, 2.1, 2.2 instead of 1, 2, 3, 4)
\numberwithin{table}{section} % Number tables within sections (i.e. 1.1, 1.2, 2.1, 2.2 instead of 1, 2, 3, 4)

\setlength\parindent{0pt} % Removes all indentation from paragraphs - comment this line for an assignment with lots of text

%----------------------------------------------------------------------------------------
%	TITLE SECTION
%----------------------------------------------------------------------------------------

\newcommand{\horrule}[1]{\rule{\linewidth}{#1}} % Create horizontal rule command with 1 argument of height

\title{	
\normalfont \normalsize 
\textsc{University of Virginia} \\ [25pt] % Your university, school and/or department name(s)
\horrule{0.5pt} \\[0.4cm] % Thin top horizontal rule
\huge ECE/CS 5565 Homework 2 \\ % The assignment title
\horrule{2pt} \\[0.5cm] % Thick bottom horizontal rule
}

\author{Shawn (Shuoshuo) Chen\\sc7cq@virginia.edu} % Your name

\date{\normalsize\today} % Today's date or a custom date

\begin{document}

\maketitle % Print the title

%----------------------------------------------------------------------------------------
%	PROBLEM 1
%----------------------------------------------------------------------------------------

\section*{Problem 1}

(a). False \\
(b). True \\
(c). False \\
(d). False \\
(e). False \\
\\


%----------------------------------------------------------------------------------------
%	PROBLEM 2
%----------------------------------------------------------------------------------------

\section*{Problem 3}
Before running HTTP, application-layer protocol DNS is needed to resolve the hostname to IP address.
And DNS is running on top of transport-layer protocol UDP. HTTP is running on top of transport-layer
protocol TCP.
\\


%----------------------------------------------------------------------------------------
%	PROBLEM 3
%----------------------------------------------------------------------------------------

\section*{Problem 4}
(a). http://gaia.cs.umass.edu/cs453/index.html  (URL=protocol+host+filenpath) \\

(b). HTTP 1.1 \\

(c). persistent (indicated by filed Connection: keep-alive) \\

(d). not able to tell by just looking at the HTTP message. But if we have the captured packet, we can check the src IP field in the IP header. \\

(e). Mozilla/5.0, the server can send different versions of the page to the client according to the browser information. \\
\\


%----------------------------------------------------------------------------------------
%	PROBLEM 4
%----------------------------------------------------------------------------------------

\section*{Problem 6}
(a). Both the server and client can signal the close of a persistent connection. Both the server and client
would check the Connection header field to see if the connection-token "close" is included. Once the
connection-token "close" is received, no further message should be sent on that connection. \\

(b). no encryption service is provided. \\

(c). The RFC suggests a single client should not maintain more than 2 connections to a server. But actually,
it really depends on the implementation of the browser vendors. \\

(d). Yes, it's possible. In a typical case described by the RFC, a client might have started to send a new request
at the same time that the server has decided to close the "idle" connection. From the server's point of view,
the connection is being closed while it was idle, but from the client's point of view, a request is in progress.\\
\\


%----------------------------------------------------------------------------------------
%	PROBLEM 5
%----------------------------------------------------------------------------------------

\section*{Problem 7}
To resolve the URL and get the IP address, the total DNS lookup time is:
\begin{align*} 
\begin{split}
T_{DNS} = \sum\limits_{i=1}^n RTT_{i} 
\end{split}					
\end{align*}
Once we have the IP address, $RTT_{0}$ is needed to set up the connection,
and another $RTT_{0}$ is needed to fetch the object. Thus, total elapsed time is:
\begin{align*} 
\begin{split}
T_{total} = \sum\limits_{i=1}^n RTT_{i} + 2 \times RTT_{0}
\end{split}					
\end{align*}
\\


%----------------------------------------------------------------------------------------
%	PROBLEM 6
%----------------------------------------------------------------------------------------

\section*{Problem 8}
(a). For non-persistent and non-parallel connection, the delay of extra 8 objects is additive:
\begin{align*} 
\begin{split}
T_{total} &= \sum\limits_{i=1}^n RTT_{i} + 2 \times RTT_{0} + 8 \times 2 \times RTT_{0} \\
&= \sum\limits_{i=1}^n RTT_{i} + 18 \times RTT_{0}
\end{split}					
\end{align*}

(b). For non-persistent but up to 5 parallel connections, the 8 objects will be fetched in 2 batches,
5 in the first request/reply and 3 in the second one. Total elapsed time is:
\begin{align*} 
\begin{split}
T_{total} &= \sum\limits_{i=1}^n RTT_{i} + 2 \times RTT_{0} + 2 \times 2 \times RTT_{0} \\
&= \sum\limits_{i=1}^n RTT_{i} + 6 \times RTT_{0}
\end{split}					
\end{align*}

(c). For persistent connections, we can do pipelining. Thus the 8 objects only cost one RTT and additive
transmission delay. However, transmission delay is ignored, so total elapsed time is:
\begin{align*} 
\begin{split}
T_{total} &= \sum\limits_{i=1}^n RTT_{i} + 2 \times RTT_{0} + RTT_{0} \\
&= \sum\limits_{i=1}^n RTT_{i} + 3 \times RTT_{0}
\end{split}					
\end{align*}
\\


%----------------------------------------------------------------------------------------
%	PROBLEM 7
%----------------------------------------------------------------------------------------

\section*{Problem 9}
(a). $\Delta$ is the transmission delay on the access link:
\begin{align*} 
\begin{split}
\Delta &= \frac{L}{R} \\
&= \frac{850,000 \  bits}{15,000,000 \  bps} \\
&= 0.0567 \quad seconds
\end{split}					
\end{align*}
The the average access delay can be calculated as:
\begin{align*} 
\begin{split}
T_{access} &= \frac{\Delta}{1-\Delta\beta} \\
&= \frac{0.0567}{1-0.0567 \times 16} \\
&= 0.61 \quad seconds
\end{split}					
\end{align*}
So the total average response time is:
\begin{align*} 
\begin{split}
T_{total} &= T_{access} + T_{Internet} \\
&= 0.61 + 3 \\
&= 3.61 \quad seconds
\end{split}					
\end{align*}

(b). Since a cache in installed in the LAN, only $40\%$ of the requests will go into the Internet,
so the $T_{access}$ now becomes:
\begin{align*} 
\begin{split}
T_{access} &= \frac{\Delta}{1-\Delta \times (0.4\beta)} \\
&= \frac{0.0567}{1-0.0567 \times (0.4 \times 16)} \\
&= 0.089 \quad seconds
\end{split}					
\end{align*}
For the missed requests, total average response time is:
\begin{align*} 
\begin{split}
T_{total1} &= T_{access} + T_{Internet} \\
&= 0.089 + 3 \\
&= 3.089 \quad seconds
\end{split}					
\end{align*}
For the hit requests, total average response time equals transmission delay because
$t_{proc}, t_{queue} and t_{prop}$ can all be ignored. Thus:
\begin{align*} 
\begin{split}
T_{total2} &= t_{trans} \\
&= \frac{L}{R} \\
&= \frac{850,000}{100,000,000} \\
&= 0.0085 \quad seconds
\end{split}					
\end{align*}
Finally, do a weighted average to get the overall average response time:
\begin{align*} 
\begin{split}
T_{total} &= 0.4 \times T_{total1} + 0.6 \times T_{total2} \\
&= 0.4 \times 3.089 + 0.6 \times 0.0085 \\
&= 1.2407 \quad seconds
\end{split}					
\end{align*}
\\


%----------------------------------------------------------------------------------------
%	PROBLEM 8
%----------------------------------------------------------------------------------------

\section*{Problem 13}
MAIL FROM in SMTP is used for the SMTP server to know whom the sender is. It needs to be a valid
email address recognizable by the SMTP server. The From filed is just a human-readable text in the message body.
\\


%----------------------------------------------------------------------------------------
%	PROBLEM 9
%----------------------------------------------------------------------------------------

\section*{Problem 22}
For client-server model, the minimum distribution time can be calculated by:
\begin{align*} 
\begin{split}
D_{cs} &= max\{ \frac{NF}{u_s}, \frac{F}{d_{min}} \}
\end{split}					
\end{align*}
For P2P model, the minimum distribution time can be calculated by:
\begin{align*} 
\begin{split}
D_{P2P} &= max\{ \frac{F}{u_s}, \frac{F}{d_{min}}, \frac{NF}{u_s + \sum \limits{i=1}^N u_i} \}
\end{split}					
\end{align*}
For client-server model:
\begin{table}[h!]
  \begin{center}
    \label{tab:table1}
    \begin{tabular}{ l | l | c | r }
      N/u & 300 Kbps & 700 Kbps & 2 Mbps \\
      \hline
      10 & 7500 sec & 7500 sec & 7500 sec \\
      \hline
      100 & 50000 sec & 50000 sec & 50000 sec \\
      \hline
      1000 & 500000 sec & 500000 sec & 500000 sec \\
    \end{tabular}
  \end{center}
\end{table}
\\
For P2P model:
\begin{table}[h!]
  \begin{center}
    \label{tab:table1}
    \begin{tabular}{ l | l | c | r }
      N/u & 300 Kbps & 700 Kbps & 2 Mbps \\
      \hline
      10 & 7500 sec & 7500 sec & 7500 sec \\
      \hline
      100 & 25000 sec & 15000 sec & 7500 sec \\
      \hline
      1000 & 45455 sec & 20548 sec & 7500 sec \\
    \end{tabular}
  \end{center}
\end{table}
\\


%----------------------------------------------------------------------------------------
%	PROBLEM 10
%----------------------------------------------------------------------------------------

\section*{Problem 23}
(a). The server sends the file to each receiver in parallel with the same rate of $\frac{u_s}{N}$.
Since $\frac{u_s}{N} \leqslant d_{min}$, each receiver can receive at the speed of $\frac{u_s}{N}$.
Thus, the distribution time is the reception time of each receiver:
\begin{align*} 
\begin{split}
D_{cs} &= \frac{F}{\frac{u_s}{N}} \\
&= \frac{NF}{u_s}
\end{split}					
\end{align*}

(b). The server sends the file to each receiver in parallel with the same rate of $d_{min}$.
Because $\frac{u_s}{N} \geqslant d_{min}$, each receiver can only receive at full speed, which is $d_{min}$.
And since all receivers are receiving in parallel, the total time equals the single reception time:
\begin{align*} 
\begin{split}
D_{cs} &= \frac{F}{d_{min}}
\end{split}					
\end{align*}

(c). For $\frac{u_s}{N} \leqslant d_{min}$, there is $D_{cs} \geqslant \frac{NF}{u_s}$.
And for $\frac{u_s}{N} \geqslant d_{min}$, there is $D_{cs} \geqslant \frac{F}{d_{min}}$.
Thus, given any $u_s$, there is $D_{cs} \geqslant max\{ \frac{NF}{u_s}, \frac{F}{d_{min}} \}$.
\\


%----------------------------------------------------------------------------------------
%	PROBLEM 11
%----------------------------------------------------------------------------------------

\section*{Problem 27}
Peer 3 will ask its first successor, peer 4, for the identifier and IP address of peer 4's immediate successor.
Since peer 5 has left, peer 4 is aware and thus makes its second successor, peer 8, its new first successor.
When peer 3 asks peer 4, it returns the information of peer 8. Peer 3 then takes peer 8 as its new second successor.
\\


%----------------------------------------------------------------------------------------
%	PROBLEM 12
%----------------------------------------------------------------------------------------

\section*{WebSocket vs HTTP}
(a). Time-sensitive applications like online games and stock tickers tend to generate data at a high frequency.
In this sense, HTTP long polling and streaming would not be very suitable because the server can only send data
upon client's request. For exmaple, in terms of HTTP long polling and streaming, the stock tickers are generated at the scale of micro seconds while network delay is usually milliseconds. When the client receives a previous response and issues a new request right away, due to network delay, quite a lot of hitorical tickers would be missed since server only returns the latest status. However, WebSocket does not have this problem because upon the update of a tick, server can push it to the client. Even though the ticks are generated at high speed, server is doing pipelining to push the data. No data will be missed. \\

(b). WebSocket and HTTP are actually unrelated. The only relationship is that websocket's handshake is sent to port 80 and interpreted by a web server as upgrade request. The difference between them is that websocket is truly full-duplex, the server can push data to the client. While HTTP is not truly full-duplex though server push can be implemented by HTTP long polling or streaming. \\

WebSocket runs on top of TCP. It establishes a single persistent TCP connection at the handshake phase.
\\


\end{document}
