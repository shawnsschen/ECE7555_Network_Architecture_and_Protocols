%%%%%%%%%%%%%%%%%%%%%%%%%%%%%%%%%%%%%%%%%
% Short Sectioned Assignment
% LaTeX Template
% Version 1.0 (5/5/12)
%
% This template has been downloaded from:
% http://www.LaTeXTemplates.com
%
% Original author:
% Frits Wenneker (http://www.howtotex.com)
%
% License:
% CC BY-NC-SA 3.0 (http://creativecommons.org/licenses/by-nc-sa/3.0/)
%
%%%%%%%%%%%%%%%%%%%%%%%%%%%%%%%%%%%%%%%%%

%----------------------------------------------------------------------------------------
%	PACKAGES AND OTHER DOCUMENT CONFIGURATIONS
%----------------------------------------------------------------------------------------

\documentclass[paper=a4, fontsize=11pt]{scrartcl} % A4 paper and 11pt font size

\usepackage[T1]{fontenc} % Use 8-bit encoding that has 256 glyphs
\usepackage{fourier} % Use the Adobe Utopia font for the document - comment this line to return to the LaTeX default
\usepackage[english]{babel} % English language/hyphenation
\usepackage{amsmath,amsfonts,amsthm} % Math packages

\usepackage{lipsum} % Used for inserting dummy 'Lorem ipsum' text into the template

\usepackage{sectsty} % Allows customizing section commands
\allsectionsfont{\centering \normalfont\scshape} % Make all sections centered, the default font and small caps

\usepackage{fancyhdr} % Custom headers and footers
\pagestyle{fancyplain} % Makes all pages in the document conform to the custom headers and footers
\fancyhead{} % No page header - if you want one, create it in the same way as the footers below
\fancyfoot[L]{} % Empty left footer
\fancyfoot[C]{} % Empty center footer
\fancyfoot[R]{\thepage} % Page numbering for right footer
\renewcommand{\headrulewidth}{0pt} % Remove header underlines
\renewcommand{\footrulewidth}{0pt} % Remove footer underlines
\setlength{\headheight}{13.6pt} % Customize the height of the header

\numberwithin{equation}{section} % Number equations within sections (i.e. 1.1, 1.2, 2.1, 2.2 instead of 1, 2, 3, 4)
\numberwithin{figure}{section} % Number figures within sections (i.e. 1.1, 1.2, 2.1, 2.2 instead of 1, 2, 3, 4)
\numberwithin{table}{section} % Number tables within sections (i.e. 1.1, 1.2, 2.1, 2.2 instead of 1, 2, 3, 4)

\setlength\parindent{0pt} % Removes all indentation from paragraphs - comment this line for an assignment with lots of text

%----------------------------------------------------------------------------------------
%	TITLE SECTION
%----------------------------------------------------------------------------------------

\newcommand{\horrule}[1]{\rule{\linewidth}{#1}} % Create horizontal rule command with 1 argument of height

\title{	
\normalfont \normalsize 
\textsc{University of Virginia} \\ [25pt] % Your university, school and/or department name(s)
\horrule{0.5pt} \\[0.4cm] % Thin top horizontal rule
\huge ECE/CS 5565 Homework 1 \\ % The assignment title
\horrule{2pt} \\[0.5cm] % Thick bottom horizontal rule
}

\author{Shawn (Shuoshuo) Chen} % Your name

\date{\normalsize\today} % Today's date or a custom date

\begin{document}

\maketitle % Print the title

%----------------------------------------------------------------------------------------
%	PROBLEM 1
%----------------------------------------------------------------------------------------

\section{Problem 2}

If assuming the sender is sending packets one after one without any gap, the packets will also arrive one after one
on the receiving side. This is a pipelining case. In addition, equation 1.1 from the textbook ignores the propagation delay on the wire.
\\

Then, for (a) 1 link, the time sender finishes sending is the time receiver finishes receiving. P packets will cost:
\begin{align}
d_{end-to-end} = P \times \frac{L}{R}
\end{align}

For (b) 2 links, at time $N \times \frac{L}{R}$, the first packet has arrived at the destination while the second one has arrived at the router. So $\frac{L}{R}$ later, the second packet will arrive at the destination.
And $(P-1) \times \frac{L}{R}$ later, the last packet arrives at the destination. The total delay for P packets is:
\begin{align} 
\begin{split}
d_{end-to-end} &= (P-1) \times \frac{L}{R} + 2 \times \frac{L}{R} \\
&= (P+1) \times \frac{L}{R}
\end{split}					
\end{align}

Similarly, for (c) N links, at time $N \times \frac{L}{R}$, the first packet has arrived at the destination while the second one is one hop away from the destination and the third one is two hops away, etc. After $\frac{L}{R}$, 
the second packet arrives at the destination and all other packets move one hop forward.
Till $(P-1) \times \frac{L}{R}$ time later, the last packet arrives at the destination. Then the total delay for P packets is:
\begin{align} 
\begin{split}
d_{end-to-end} &= (P-1) \times \frac{L}{R} + N \times \frac{L}{R} \\
&= (P+N-1) \times \frac{L}{R}
\end{split}					
\end{align}



%----------------------------------------------------------------------------------------
%	PROBLEM 2
%----------------------------------------------------------------------------------------

\section{Problem 5}


\end{document}