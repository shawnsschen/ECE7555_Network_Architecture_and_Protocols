%%%%%%%%%%%%%%%%%%%%%%%%%%%%%%%%%%%%%%%%%
% Short Sectioned Assignment
% LaTeX Template
% Version 1.0 (5/5/12)
%
% This template has been downloaded from:
% http://www.LaTeXTemplates.com
%
% Original author:
% Frits Wenneker (http://www.howtotex.com)
%
% License:
% CC BY-NC-SA 3.0 (http://creativecommons.org/licenses/by-nc-sa/3.0/)
%
%%%%%%%%%%%%%%%%%%%%%%%%%%%%%%%%%%%%%%%%%

%----------------------------------------------------------------------------------------
%	PACKAGES AND OTHER DOCUMENT CONFIGURATIONS
%----------------------------------------------------------------------------------------

\documentclass[titlepage, paper=a4, fontsize=11pt]{scrartcl} % A4 paper and 11pt font size

\usepackage[T1]{fontenc} % Use 8-bit encoding that has 256 glyphs
\usepackage{fourier} % Use the Adobe Utopia font for the document - comment this line to return to the LaTeX default
\usepackage[english]{babel} % English language/hyphenation
\usepackage{amsmath,amsfonts,amsthm} % Math packages
\usepackage{listings}

\usepackage{lipsum} % Used for inserting dummy 'Lorem ipsum' text into the template
\usepackage{graphicx}
\usepackage{caption}


\usepackage{sectsty} % Allows customizing section commands
\allsectionsfont{\centering \normalfont\scshape} % Make all sections centered, the default font and small caps

\usepackage{fancyhdr} % Custom headers and footers
\pagestyle{fancyplain} % Makes all pages in the document conform to the custom headers and footers
\fancyhead{} % No page header - if you want one, create it in the same way as the footers below
\fancyfoot[L]{} % Empty left footer
\fancyfoot[C]{} % Empty center footer
\fancyfoot[R]{\thepage} % Page numbering for right footer
\renewcommand{\headrulewidth}{0pt} % Remove header underlines
\renewcommand{\footrulewidth}{0pt} % Remove footer underlines
\setlength{\headheight}{13.6pt} % Customize the height of the header

\numberwithin{equation}{section} % Number equations within sections (i.e. 1.1, 1.2, 2.1, 2.2 instead of 1, 2, 3, 4)
\numberwithin{figure}{section} % Number figures within sections (i.e. 1.1, 1.2, 2.1, 2.2 instead of 1, 2, 3, 4)
\numberwithin{table}{section} % Number tables within sections (i.e. 1.1, 1.2, 2.1, 2.2 instead of 1, 2, 3, 4)

\setlength\parindent{0pt} % Removes all indentation from paragraphs - comment this line for an assignment with lots of text

%----------------------------------------------------------------------------------------
%	TITLE SECTION
%----------------------------------------------------------------------------------------

\newcommand{\horrule}[1]{\rule{\linewidth}{#1}} % Create horizontal rule command with 1 argument of height

\title{	
\normalfont \normalsize 
\textsc{University of Virginia} \\ [25pt] % Your university, school and/or department name(s)
\horrule{0.5pt} \\[0.4cm] % Thin top horizontal rule
\huge ECE/CS 5565 Project 1 \\ % The assignment title
\horrule{2pt} \\[0.5cm] % Thick bottom horizontal rule
}

\renewcommand{\thefigure}{\arabic{figure}}

\author{Shawn (Shuoshuo) Chen\\sc7cq@virginia.edu} % Your name

\date{\normalsize\today} % Today's date or a custom date

\begin{document}

\maketitle % Print the title

%----------------------------------------------------------------------------------------
%	PROBLEM 1
%----------------------------------------------------------------------------------------

\section*{\textbf{Lab 1}}
\subsection*{Question 1}
DHCP, DNS, ARP
\\

\subsection*{Question 2}
HTTP GET was sent at 22:34:56.094124999 and the HTTP OK was received at 22:34:56.115811000.
It took about 0.02 seconds (20 milliseconds).
\\

\subsection*{Question 3}
The Internet address of gaia.cs.umass.edu is $128.119.245.12$. The Internet address of my
computer is $128.143.137.117$.
\\

\subsection*{Question 4}
See Figure ~\ref{fig:http}.
\\[12pt]


\section*{\textbf{Lab 2}}
\begin{figure}[!ht]
    \centering
    \includegraphics[width=\textwidth]{images/http-cap.pdf}
    \caption{HTTP GET and response}
    \label{fig:http}
\end{figure}

\subsection*{Question 1}
My browser is running HTTP 1.1. It can be found from the GET packet:
\begin{verbatim}
GET /wireshark-labs/HTTP-wireshark-file1.html HTTP/1.1\r\n
\end{verbatim}

The server is running HTTP 1.1. It can be found from the response packet:
\begin{verbatim}
HTTP/1.1 200 OK\r\n
\end{verbatim}


\subsection*{Question 2}
English. It can be found from the GET packet:
\begin{verbatim}
Accept-Language: en-US,en;q=0.8\r\n
\end{verbatim}


\subsection*{Question 3}
IP of my computer is 128.143.137.117, IP of gaia.cs.umass.edu is 128.119.245.12.
This can be found from the IP header of the GET packet.
\begin{verbatim}
Internet Protocol Version 4, Src: 128.143.137.117 (128.143.137.117),
Dst: 128.119.245.12 (128.119.245.12)
\end{verbatim}


\subsection*{Question 4}
Status code is 200. This is found from the response packet:
\begin{verbatim}
HTTP/1.1 200 OK\r\n
\end{verbatim}


\subsection*{Question 5}
It was last modified at 05:59:01 GMT on Oct. 8th, 2015. This is found from the response packet:
\begin{verbatim}
Last-Modified: Thu, 08 Oct 2015 05:59:01 GMT\r\n
\end{verbatim}


\subsection*{Question 6}
The content length is 128 bytes, found from the response packet:
\begin{verbatim}
Content-Length: 128\r\n
\end{verbatim}


\subsection*{Question 7}
Yes. Within the response body, there is a `Line-based text data: text/html' filed.
If expand the field, we will see the source code of that page inside:
\begin{verbatim}
<html>\n
Congratulations.  You've downloaded the file \n
http://gaia.cs.umass.edu/wireshark-labs/HTTP-wireshark-file1.html!\n
</html>\n
\end{verbatim}

\end{document}
