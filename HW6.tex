%%%%%%%%%%%%%%%%%%%%%%%%%%%%%%%%%%%%%%%%%
% Short Sectioned Assignment
% LaTeX Template
% Version 1.0 (5/5/12)
%
% This template has been downloaded from:
% http://www.LaTeXTemplates.com
%
% Original author:
% Frits Wenneker (http://www.howtotex.com)
%
% License:
% CC BY-NC-SA 3.0 (http://creativecommons.org/licenses/by-nc-sa/3.0/)
%
%%%%%%%%%%%%%%%%%%%%%%%%%%%%%%%%%%%%%%%%%

%----------------------------------------------------------------------------------------
%	PACKAGES AND OTHER DOCUMENT CONFIGURATIONS
%----------------------------------------------------------------------------------------

\documentclass[titlepage, paper=a4, fontsize=11pt]{scrartcl} % A4 paper and 11pt font size

\usepackage[T1]{fontenc} % Use 8-bit encoding that has 256 glyphs
\usepackage{fourier} % Use the Adobe Utopia font for the document - comment this line to return to the LaTeX default
\usepackage[english]{babel} % English language/hyphenation
\usepackage{amsmath,amsfonts,amsthm} % Math packages
\usepackage{listings}

\usepackage{lipsum} % Used for inserting dummy 'Lorem ipsum' text into the template
\usepackage{graphicx}


\usepackage{sectsty} % Allows customizing section commands
\allsectionsfont{\centering \normalfont\scshape} % Make all sections centered, the default font and small caps

\usepackage{fancyhdr} % Custom headers and footers
\pagestyle{fancyplain} % Makes all pages in the document conform to the custom headers and footers
\fancyhead{} % No page header - if you want one, create it in the same way as the footers below
\fancyfoot[L]{} % Empty left footer
\fancyfoot[C]{} % Empty center footer
\fancyfoot[R]{\thepage} % Page numbering for right footer
\renewcommand{\headrulewidth}{0pt} % Remove header underlines
\renewcommand{\footrulewidth}{0pt} % Remove footer underlines
\setlength{\headheight}{13.6pt} % Customize the height of the header

\numberwithin{equation}{section} % Number equations within sections (i.e. 1.1, 1.2, 2.1, 2.2 instead of 1, 2, 3, 4)
\numberwithin{table}{section} % Number tables within sections (i.e. 1.1, 1.2, 2.1, 2.2 instead of 1, 2, 3, 4)

\setlength\parindent{0pt} % Removes all indentation from paragraphs - comment this line for an assignment with lots of text

%----------------------------------------------------------------------------------------
%	TITLE SECTION
%----------------------------------------------------------------------------------------

\newcommand{\horrule}[1]{\rule{\linewidth}{#1}} % Create horizontal rule command with 1 argument of height

\title{	
\normalfont \normalsize 
\textsc{University of Virginia} \\ [25pt] % Your university, school and/or department name(s)
\horrule{0.5pt} \\[0.4cm] % Thin top horizontal rule
\huge ECE/CS 5565 Homework 6 \\ % The assignment title
\horrule{2pt} \\[0.5cm] % Thick bottom horizontal rule
}

\author{Shawn (Shuoshuo) Chen\\sc7cq@virginia.edu} % Your name

\date{\normalsize\today} % Today's date or a custom date

\begin{document}

\maketitle % Print the title

%----------------------------------------------------------------------------------------
%	PROBLEM 1
%----------------------------------------------------------------------------------------

\section*{Problem 1}
The two-dimentional even parity matrix looks like:
\begin{tabular}{ r | r | r | r }
  1\ 1\ 1\ 0 & 1 \\
  0\ 1\ 1\ 0 & 0 \\
  1\ 0\ 0\ 1 & 0 \\
  1\ 1\ 0\ 1 & 1 \\
  \hline
  1\ 1\ 0\ 0 & 0 \\
\end{tabular}
\\



%----------------------------------------------------------------------------------------
%	PROBLEM 5
%----------------------------------------------------------------------------------------

\section*{Problem 5}
Since G=10011, we know r=4. Thus $D*2^r=10101010100000$. Then we can compute R by:
\begin{align*} 
\begin{split}
R&=remainder(\frac{D*2^r}{G}) \\
&= remainder(\frac{10101010100000}{10011}) \\
&= 0100
\end{split}					
\end{align*}
\\



%----------------------------------------------------------------------------------------
%	PROBLEM 10
%----------------------------------------------------------------------------------------

\section*{Problem 10}
(a).
Assume the bandwidth of the link is $R$, A's average throughput is $p_A(1-p_B)*R$. \\
Total effeciency is $p_A(1-p_B)+p_B(1-p_A)$. \\

(b). No. If $p_A=2p_B$, A's average throughput is $2p_B(1-p_B)*R$. B's average throughput is
$p_B(1-2p_B)$. A's throughput is not twice of B's. \\
To make that happen, there should be $p_A(1-p_B)=2p_B(1-p_A)$. Thus, there should be $p_B=\frac{p_A}{2-p_A}$. \\

(c). A's average throughput can be given by $2p(1-p)^{N-1}*R$. Other nodes have the same average throughput of $p(1-2p)(1-p)^{N-2}*R$. \\




%----------------------------------------------------------------------------------------
%	PROBLEM 14
%----------------------------------------------------------------------------------------

\section*{Problem 14}
(a). \\
A: 192.168.1.2 \\
B: 192.168.1.3 \\
Interface of left router connecting subnet 1: 192.168.1.1 \\
C: 192.168.2.2 \\
D: 192.168.2.3 \\
Interface of left router connecting subnet 2: 192.168.2.1 \\
Interface of right router connecting subnet 2: 192.168.2.4 \\
E: 192.168.3.2 \\
F: 192.168.3.3 \\
Interface of right router connecting subnet 3: 192.168.3.1 \\

(b). \\
A: 00-00-00-00-00-01 \\
B: 00-00-00-00-00-02 \\
Interface of left router connecting subnet 1: 00-00-00-00-00-03 \\
C: 00-00-00-00-00-04 \\
D: 00-00-00-00-00-05 \\
Interface of left router connecting subnet 2: 00-00-00-00-00-06 \\
Interface of right router connecting subnet 2: 00-00-00-00-00-07 \\
E: 00-00-00-00-00-08 \\
F: 00-00-00-00-00-09 \\
Interface of right router connecting subnet 3: 00-00-00-00-00-10 \\

(c). \\
1. routing table in E directs the datagram via interface 192.168.3.2 \\
2. adapter in E creates Ethernet frame with Ethernet destination address 00-00-00-00-00-10 \\
3. right router receives the packet and extracts the datagram. forwarding table in this router routes the datagram to 192.168.2.1 \\
4. right router then sends the Ethernet frame with the destination address of 00-00-00-00-00-06 and source address of 00-00-00-00-00-07 via its interface with IP address of 198.162.2.4 \\
5. left router receives the datagram and repeats the same process (3,4) and eventually sends to B. \\

(d). \\
Before starting the process in (c), E first needs to update its ARP table. It first sends out an ARP query via broadcast. Right router responds with an ARP response indicating the desired MAC address is 00-00-00-00-00-10. Then E goes over the steps in (c).




%----------------------------------------------------------------------------------------
%	PROBLEM 15
%----------------------------------------------------------------------------------------

\section*{Problem 15}
(a). No, because they are in the same subnet. Source IP is E's IP, destination IP is F's IP. Source MAC is E's MAC, destination MAC is F's MAC. \\

(b). No, because E and B are not in the same subnet. Source IP is E's IP, destination IP is B's IP. Source MAC is E's MAC, destination MAC is the MAC of the interface of R1 that connects to subnet 3. \\

(c). S1 will broadcast the ARP to all interfaces except for the incoming interface. It also learns A's MAC and its connected interface. It updates its forwarding table and adds an entry for A. \\
Yes, R1 also receives the ARP. No, R1 will not forward the ARP. \\
No, because the ARP query already contains A's MAC. \\
S1 learns B's MAC and interface, thus adds an entry for B. But then it discards all frames between A and B because they are from the same interface from the perspective of S1.
\\



%----------------------------------------------------------------------------------------
%	PROBLEM 19
%----------------------------------------------------------------------------------------

\section*{Problem 19}
Since $K_A=0, K_B=1$, A will retransmit immediately while B will wait. The time that B waits is $K_B*512$ bit times, which is $K_B*512*1/10Mbps=51.2$ us. Considering the propagation of the signal, it takes $245*1/10Mbps=24.5$ us to collide and $48*1/10Mbps=4.8$ us to send the jam signal. So B shcedules its retransmission at $t=80.5$ us. But B needs to sense an idle channel before sending. \\

A receives the collision signal after 245 bits time and transmits the jam frame for 48 bits time. Then it needs to wait for another 245 bits time to wait for the last bit from B to leav the wire. Finally, it waits for 96 bits time as the minimum inter-frame gap to start sending. So A starts to transmit at $t=(245+48+245+96)*1/10Mbps=63.4$ us. \\

At $t=63.4+245*1/10Mbps=87.9$ us, A's signal reaches B. \\

Yes, B refrains because at $t=80.5$ us, B senses A's signal.



\end{document}
