%%%%%%%%%%%%%%%%%%%%%%%%%%%%%%%%%%%%%%%%%
% Short Sectioned Assignment
% LaTeX Template
% Version 1.0 (5/5/12)
%
% This template has been downloaded from:
% http://www.LaTeXTemplates.com
%
% Original author:
% Frits Wenneker (http://www.howtotex.com)
%
% License:
% CC BY-NC-SA 3.0 (http://creativecommons.org/licenses/by-nc-sa/3.0/)
%
%%%%%%%%%%%%%%%%%%%%%%%%%%%%%%%%%%%%%%%%%

%----------------------------------------------------------------------------------------
%	PACKAGES AND OTHER DOCUMENT CONFIGURATIONS
%----------------------------------------------------------------------------------------

\documentclass[titlepage, paper=a4, fontsize=11pt]{scrartcl} % A4 paper and 11pt font size

\usepackage[T1]{fontenc} % Use 8-bit encoding that has 256 glyphs
\usepackage{fourier} % Use the Adobe Utopia font for the document - comment this line to return to the LaTeX default
\usepackage[english]{babel} % English language/hyphenation
\usepackage{amsmath,amsfonts,amsthm} % Math packages
\usepackage{listings}

\usepackage{lipsum} % Used for inserting dummy 'Lorem ipsum' text into the template
\usepackage{graphicx}


\usepackage{sectsty} % Allows customizing section commands
\allsectionsfont{\centering \normalfont\scshape} % Make all sections centered, the default font and small caps

\usepackage{fancyhdr} % Custom headers and footers
\pagestyle{fancyplain} % Makes all pages in the document conform to the custom headers and footers
\fancyhead{} % No page header - if you want one, create it in the same way as the footers below
\fancyfoot[L]{} % Empty left footer
\fancyfoot[C]{} % Empty center footer
\fancyfoot[R]{\thepage} % Page numbering for right footer
\renewcommand{\headrulewidth}{0pt} % Remove header underlines
\renewcommand{\footrulewidth}{0pt} % Remove footer underlines
\setlength{\headheight}{13.6pt} % Customize the height of the header

\numberwithin{equation}{section} % Number equations within sections (i.e. 1.1, 1.2, 2.1, 2.2 instead of 1, 2, 3, 4)
\numberwithin{table}{section} % Number tables within sections (i.e. 1.1, 1.2, 2.1, 2.2 instead of 1, 2, 3, 4)

\setlength\parindent{0pt} % Removes all indentation from paragraphs - comment this line for an assignment with lots of text

%----------------------------------------------------------------------------------------
%	TITLE SECTION
%----------------------------------------------------------------------------------------

\newcommand{\horrule}[1]{\rule{\linewidth}{#1}} % Create horizontal rule command with 1 argument of height

\title{	
\normalfont \normalsize 
\textsc{University of Virginia} \\ [25pt] % Your university, school and/or department name(s)
\horrule{0.5pt} \\[0.4cm] % Thin top horizontal rule
\huge ECE/CS 5565 Homework 5 \\ % The assignment title
\horrule{2pt} \\[0.5cm] % Thick bottom horizontal rule
}

\author{Shawn (Shuoshuo) Chen\\sc7cq@virginia.edu} % Your name

\date{\normalsize\today} % Today's date or a custom date

\begin{document}

\maketitle % Print the title

%----------------------------------------------------------------------------------------
%	PROBLEM 4
%----------------------------------------------------------------------------------------

\section*{Problem 4}
(a).
The partial forwarding table in router A: \\
\begin{tabular}{ r | r }
  Destination Address & Output Interface \\
  H3 & 3 \\
\end{tabular}
\\

(b).
In terms of a conventional router, it is impossible. Because the router only inspects the destination address field in a packet, in which case, H1-sourced packets and H2-sourced packets contain the same dst addr. However, newer technologies like OpenFlow can do this by checking the source address and form a more complicated rule in the forwarding table.
\\

(c).
The forwarding table for VC looks lile: \\
\begin{tabular}{ r | r | r | r }
  Incoming Interface & Incoming VC \# & Outgoing Interface & Outgoing VC \# \\
  1 & 10 & 3 & 30 \\
  2 & 20 & 4 & 40 \\
\end{tabular}
\\

(d).
For router B: \\
\begin{tabular}{ r | r | r | r }
  Incoming Interface & Incoming VC \# & Outgoing Interface & Outgoing VC \# \\
  1 & 30 & 2 & 31 \\
\end{tabular}
\\
For router C: \\
\begin{tabular}{ r | r | r | r }
  Incoming Interface & Incoming VC \# & Outgoing Interface & Outgoing VC \# \\
  1 & 40 & 2 & 41 \\
\end{tabular}
\\
For router D: \\
\begin{tabular}{ r | r | r | r }
  Incoming Interface & Incoming VC \# & Outgoing Interface & Outgoing VC \# \\
  1 & 31 & 3 & 50 \\
  2 & 41 & 3 & 51 \\
\end{tabular}
\\


%----------------------------------------------------------------------------------------
%	PROBLEM 7
%----------------------------------------------------------------------------------------

\section*{Problem 7}
(a).
No, only one packet can use the bus at the same time. The shared bus is exclusive. \\

(b).
Yes, as long as the two packets travel along two different paths and not intervene each other, this can be done in parallel. \\

(c).
No, since they are destined for the same output port, part of the path is exclusive. The switching fabric can only serve one packet at a time. \\



%----------------------------------------------------------------------------------------
%	PROBLEM 10
%----------------------------------------------------------------------------------------

\section*{Problem 10}


\end{document}
